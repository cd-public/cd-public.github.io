% Don't touch this %%%%%%%%%%%%%%%%%%%%%%%%%%%%%%%%%%%%%%%%%%%
\documentclass[11pt]{article}
\usepackage{fullpage}
\usepackage[left=1in,top=1in,right=1in,bottom=1in,headheight=3ex,headsep=3ex]{geometry}
\usepackage{graphicx}
\usepackage{float}

\newcommand{\blankline}{\quad\pagebreak[2]}
%%%%%%%%%%%%%%%%%%%%%%%%%%%%%%%%%%%%%%%%%%%%%%%%%%%%%%%%%%%%%%

% Modify Course title, instructor name, semester here %%%%%%%%

\title{CS-241: Data Structures}
\author{Calvin Deutschbein (they)}
\date{Willamette University, Fall 2021}

%%%%%%%%%%%%%%%%%%%%%%%%%%%%%%%%%%%%%%%%%%%%%%%%%%%%%%%%%%%%%%

% Don't touch this %%%%%%%%%%%%%%%%%%%%%%%%%%%%%%%%%%%%%%%%%%%
\usepackage[sc]{mathpazo}
\linespread{1.05} % Palatino needs more leading (space between lines)
\usepackage[T1]{fontenc}
\usepackage[mmddyyyy]{datetime}% http://ctan.org/pkg/datetime
\usepackage{advdate}% http://ctan.org/pkg/advdate
\newdateformat{syldate}{\twodigit{\THEMONTH}/\twodigit{\THEDAY}}
\newsavebox{\MONDAY}\savebox{\MONDAY}{Mon}% Mon
\newcommand{\week}[1]{%
%  \cleardate{mydate}% Clear date
% \newdate{mydate}{\the\day}{\the\month}{\the\year}% Store date
  \paragraph*{\kern-2ex\quad #1, \syldate{\today} - \AdvanceDate[4]\syldate{\today}:}% Set heading  \quad #1
%  \setbox1=\hbox{\shortdayofweekname{\getdateday{mydate}}{\getdatemonth{mydate}}{\getdateyear{mydate}}}%
  \ifdim\wd1=\wd\MONDAY
    \AdvanceDate[7]
  \else
    \AdvanceDate[7]
  \fi%
}
\usepackage{setspace}
\usepackage{multicol}
%\usepackage{indentfirst}
\usepackage{fancyhdr,lastpage}
\usepackage{url}
\pagestyle{fancy}
\usepackage{hyperref}
\usepackage{lastpage}
\usepackage{amsmath}
\usepackage{layout}

\lhead{}
\chead{}
%%%%%%%%%%%%%%%%%%%%%%%%%%%%%%%%%%%%%%%%%%%%%%%%%%%%%%%%%%%%%%

% Modify header here %%%%%%%%%%%%%%%%%%%%%%%%%%%%%%%%%%%%%%%%%
\rhead{\footnotesize Data Structures}

%%%%%%%%%%%%%%%%%%%%%%%%%%%%%%%%%%%%%%%%%%%%%%%%%%%%%%%%%%%%%%
% Don't touch this %%%%%%%%%%%%%%%%%%%%%%%%%%%%%%%%%%%%%%%%%%%
\lfoot{}
\cfoot{\small \thepage/\pageref*{LastPage}}
\rfoot{}

\usepackage{array, xcolor}
\usepackage{color,hyperref}
\definecolor{clemsonorange}{HTML}{EA6A20}
\hypersetup{colorlinks,breaklinks,linkcolor=clemsonorange,urlcolor=clemsonorange,anchorcolor=clemsonorange,citecolor=black}

\begin{document}

\maketitle

\blankline

\begin{tabular*}{.93\textwidth}{@{\extracolsep{\fill}}lr}

%%%%%%%%%%%%%%%%%%%%%%%%%%%%%%%%%%%%%%%%%%%%%%%%%%%%%%%%%%%%%%

% Modify information %%%%%%%%%%%%%%%%%%%%%%%%%%%%%%%%%%%%%%%%%
E-mail: \texttt{ckdeutschbein@willamette.edu} & Web: \href{https://cd-public.github.io/courses/data_structures/241f21.html}{\tt\bf cd-public.github.io/}  \\

 Office Hours: By Appt. \& TBA  &  Lecture MWF 8:00-9:00 AM \\

 Office: Zoom, Ford 206 & Lecture Hall: Ford 204 \\
 & \\
Lab Room: Ford 202 & Lab Hours: MWF 9:10-10:10 AM \\
&  \\
\hline
\end{tabular*}

\vspace{5 mm}

% First Section %%%%%%%%%%%%%%%%%%%%%%%%%%%%%%%%%%%%%%%%%%%%

\section*{Course Description}

Theoretical and practical study of programming and abstract data types including lists, stacks, queues, trees and algorithms used on these data structures. The course includes object implementation of structures and sharpens programming skills learned in previous courses.

% Second Section %%%%%%%%%%%%%%%%%%%%%%%%%%%%%%%%%%%%%%%%%%%

\section*{Required Materials}

Required materials will be available on the course webpage under \href{https://cd-public.github.io/courses/data_structures/r-241f21.html}{resources}.


% Third Section %%%%%%%%%%%%%%%%%%%%%%%%%%%%%%%%%%%%%%%%%%%

\section*{Prerequisites}

This class is meant for computer science students who have completed introductory coursework in programming. The prerequisites are CS 141 or CS 151 (though CS 151 is recommended as the course will primarily use Python).

\section*{Accessability}

I will make every effort to ensure all coursework and materials are accessible to all students, including working with on-campus specialists. However, there is always room for improvement. I always appreciate hearing from students about how I can make the course more accessible, so please reach out if there is something I can be doing better!

% Fourth Section %%%%%%%%%%%%%%%%%%%%%%%%%%%%%%%%%%%%%%%%%%%

\section*{Course Objectives}
This course will teach you how to organize the data used in computer programs so that manipulation of that data can be done efficiently on large problems and large data instances.  This course will address both how to use the data structures found in the libraries of programming languages, and how those libraries are constructed and why the items that are included in them are there (and why some are excluded). 
\begin{itemize}
\item You will gain familiarity with important categories of problems that are commonly encountered in software development, and will learn what data organizations will allow practical and efficient solutions to those problems. 
\item You will learn the basics of how to describe and analyze the performance and efficiency of your algorithms; this will prepare you for the upper level course in algorithms. 
\item You will demonstrate the concepts you learn by encoding them in correct Python programs. 
\item You will gain more proficiency in basic programming and in constructing larger programs than you have been doing in your intro classes. 
\item You will practice the induction proof methods to reason about your programs, and to argue in support of them.  
\end{itemize}
Competency of the data structures taught in this course will prepare you to study higher level areas where those data structures are heavily used: operating systems, networking, graphics, vision, compilers, databases, and security.

% Fifth Section %%%%%%%%%%%%%%%%%%%%%%%%%%%%%%%%%%%%%%%%%%%

\section*{Course Structure}

The course will be composed of lectures, labs, homeworks, midterms, and a final project.

\subsection*{Lecture Structure}

Lectures are scheduled for Mondays, Wednesdays, and Fridays at 8:00 AM in Ford 204. The schedule
of lectures will be available on the course webpage under \href{https://cd-public.github.io/courses/data_structures/s-241f21.html}{schedule}.

\bigskip
\noindent Lectures will be primarily on the white board, with some coding. I will support a remote option
for all lectures, likely through a streaming platform or Zoom, and will post recordings of lectures
on the course website. While I will make every effort to follow best practices for accessible teaching, I will make mistakes! Please, if you find some material is inaccessible for any reason
do not hesitate to reach out.

\subsection*{Lab Structure}

In addition to lecture, you must also sign up for the associated laboratory component of the
class, which would ordinarily meet in Ford 202 in the scheduling slot immediately
following the lecture (9:10 to 10:10 AM). I will follow the established practices
for operating labratory sessions:
\begin{itemize}
\item The labs will be open during the scheduled hours, so that you can use the lab computers
to complete the assignments.
\item I will be available via Zoom or in person in my office in Ford 206 during the labs to answer
questions and help you when you get stuck.
\item You may do the lab work on your own computer and need not attend the lab sessions at
the scheduled times.
\item On weeks when assignments or projects or not due, I reserve the right to assign labs. You are responsible for submitting the the required materials for a lab session by the deadline at
11:10 AM on Fridays.
\item There will be no lab assignments during the weeks that assignments are due, to encourage
you to use lab sessions to work on the projects.
\end{itemize}

\subsection*{Homework Structure}

Feedback will be provided on homework assignments. Homework assignments will be considered when determining grades for this course.

\bigskip
\noindent Homeworks will consist of programming assignments to be completed outside of class and submitted
for feedback. There will be a ``Homework 0'' at the beginning of the semester to get used to
programming and assignment submission, then four homework assignments throughout the semester.
You will always have at least two weeks to complete homework assignments.

\bigskip
\noindent Homework assignment will be due at 8:00 AM on Fridays so we can discuss them in class on the 
day they are due.

\bigskip
\noindent As a rule, I encourage students to submit their assignment as-is at the due date and not to
submit late work. By way of explanation, it is my experience as both a student and a grader that time spent working on assignments after their deadline is often better spent working on the next assignment.

\bigskip
\noindent As-is submission is supported in the grading policy by dropping the lowest homework
grade. In special circumstances such as extended medical problems or other unforeseeable
emergencies, please reach out and so we can collaboratively develop a more personal
solution to achieve the learning objectives of the course.

\subsubsection*{Midterm Structure}

Feedback will be provided on midterm exams.  Midterm exams will be considered when determining grades for this course.

\bigskip
\noindent There will be two written midterm final exams, intended to be completed individual
without access to notes or documentation. The midterms are intended to achieve a learning
focus of reasoning about data structures in isolation from coding environments,
as well provide me as an instructor with greater insight into how effective course
instruction has been.

\subsubsection*{Final Project}

Feedback will be provided on the final project. The final project will be considered when determining grades for this course.


There will be a final project that will be similar in format but distinct in content from
homeworks. Whereas homeworks will including programming assignments for which there is
some correct answer that may be measured against an answer key and notions of coding
style, the project will be a student-centered exploration of data structures meant to
give you an opportunity to apply what you've learned in the course while still receiving
support from an instructor.

\bigskip
\noindent The final project will be released as soon as the final homework is due,
on November 12, and will be due at the time of the final examination as set 
by the registrar.

\subsection*{Feedback and Grading}
Feedback will be provided on assignments, midterms, and the final project using a 100 point scale.
This 100 point scale is intended to be familiar to established grading standards, such as letter grades. To provide aggregate feedback for the whole course, these feedback scores will be combined as follows:
\begin{itemize}
	\item \underline{\textbf{40\%}} of your grade will be determined by homework assignments
	\subitem \underline{\textbf{10\%}} each for the four highest scored assignments (out of five).
	\item \underline{\textbf{40\%}} of your grade will be determined by midterm exams.
	\subitem \underline{\textbf{20\%}} each for the two midterms.
	\item \underline{\textbf{20\%}} of your grade will be determined by the final project.
\end{itemize}

\noindent Feedback scores will constitute the minimum grade on an assignment, but the instructor
may exercise discretion at any time to award a higher grade. For example, a submitted homework
may not use some important algorithmic technique as submitted, but if the student showed familiarity
with this technique on an earlier assignment or exam, the absence of that technique in a specific
case need not be counted against a student in grading, but only noted in feedback. This corresponds to the high level notion of feedback corresponding to how well an assignment reached the intended learning goals, while the overall course grade is meant to indicate that a student is prepared to succeed in latter coursework. Under this model, the final project will offer an opportunity to show familiarity with all content in the course, so a strong final project can ensure a high course grade
for any student, regardless of prior scores on midterms and homeworks.

\end{document}

