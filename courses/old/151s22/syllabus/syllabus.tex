% Don't touch this %%%%%%%%%%%%%%%%%%%%%%%%%%%%%%%%%%%%%%%%%%%
\documentclass[11pt]{article}
\usepackage{fullpage}
\usepackage[left=1in,top=1in,right=1in,bottom=1in,headheight=3ex,headsep=3ex]{geometry}
\usepackage{graphicx}
\usepackage{float}
\usepackage{quoting}

\setlength{\parindent}{0em}
\setlength{\parskip}{1em}

\newcommand{\blankline}{\quad\pagebreak[2]}
%%%%%%%%%%%%%%%%%%%%%%%%%%%%%%%%%%%%%%%%%%%%%%%%%%%%%%%%%%%%%%

% Modify Course title, instructor name, semester here %%%%%%%%

\title{CS-151: Introduction to Programming in Python}
\author{Calvin Deutschbein (they)}
\date{Willamette University, Spring 2022}

%%%%%%%%%%%%%%%%%%%%%%%%%%%%%%%%%%%%%%%%%%%%%%%%%%%%%%%%%%%%%%

% Don't touch this %%%%%%%%%%%%%%%%%%%%%%%%%%%%%%%%%%%%%%%%%%%
\usepackage[sc]{mathpazo}
\linespread{1.05} % Palatino needs more leading (space between lines)
\usepackage[T1]{fontenc}
\usepackage[mmddyyyy]{datetime}% http://ctan.org/pkg/datetime
\usepackage{advdate}% http://ctan.org/pkg/advdate
\newdateformat{syldate}{\twodigit{\THEMONTH}/\twodigit{\THEDAY}}
\newsavebox{\MONDAY}\savebox{\MONDAY}{Mon}% Mon
\newcommand{\week}[1]{%
%  \cleardate{mydate}% Clear date
% \newdate{mydate}{\the\day}{\the\month}{\the\year}% Store date
  \paragraph*{\kern-2ex\quad #1, \syldate{\today} - \AdvanceDate[4]\syldate{\today}:}% Set heading  \quad #1
%  \setbox1=\hbox{\shortdayofweekname{\getdateday{mydate}}{\getdatemonth{mydate}}{\getdateyear{mydate}}}%
  \ifdim\wd1=\wd\MONDAY
    \AdvanceDate[7]
  \else
    \AdvanceDate[7]
  \fi%
}
\usepackage{setspace}
\usepackage{multicol}
%\usepackage{indentfirst}
\usepackage{fancyhdr,lastpage}
\usepackage{url}
\pagestyle{fancy}
\usepackage{hyperref}
\usepackage{lastpage}
\usepackage{amsmath}
\usepackage{layout}

\lhead{}
\chead{}
%%%%%%%%%%%%%%%%%%%%%%%%%%%%%%%%%%%%%%%%%%%%%%%%%%%%%%%%%%%%%%

% Modify header here %%%%%%%%%%%%%%%%%%%%%%%%%%%%%%%%%%%%%%%%%
\rhead{\footnotesize Intro}

%%%%%%%%%%%%%%%%%%%%%%%%%%%%%%%%%%%%%%%%%%%%%%%%%%%%%%%%%%%%%%
% Don't touch this %%%%%%%%%%%%%%%%%%%%%%%%%%%%%%%%%%%%%%%%%%%
\lfoot{}
\cfoot{\small \thepage/\pageref*{LastPage}}
\rfoot{}

\usepackage{array, xcolor}
\usepackage{color,hyperref}
\definecolor{clemsonorange}{HTML}{EA6A20}
\hypersetup{colorlinks,breaklinks,linkcolor=clemsonorange,urlcolor=clemsonorange,anchorcolor=clemsonorange,citecolor=black}

\begin{document}

\maketitle

\blankline

\begin{tabular*}{.93\textwidth}{@{\extracolsep{\fill}}lr}

%%%%%%%%%%%%%%%%%%%%%%%%%%%%%%%%%%%%%%%%%%%%%%%%%%%%%%%%%%%%%%

% Modify information %%%%%%%%%%%%%%%%%%%%%%%%%%%%%%%%%%%%%%%%%
E-mail: \texttt{ckdeutschbein@willamette.edu} & Web: \href{https://cd-public.github.io/courses/intro/151s22.html}{\tt\bf cd-public.github.io/}  \\

 Office Hours: By Appt. \& TBA  &  Lecture MWF 9:10-10:10 AM \\

 Office: Zoom, Ford 206 & Lecture Hall: "Salem TBA" \\
 & \\
&  \\
\hline
\end{tabular*}

\vspace{5 mm}

% First Section %%%%%%%%%%%%%%%%%%%%%%%%%%%%%%%%%%%%%%%%%%%%

\section*{Course Description}

An introduction to computer science using Python. Introduces students to the fundamental concepts of programming and computational problem solving. Students will study and create programs that perform various tasks, including text and file manipulation, internet scraping, data structures, and testing. Topics will include general programming idioms such as variables, logic and loops as well as Python specific idioms such as list comprehension and generators. Object-oriented programming will be introduced. 

% Second Section %%%%%%%%%%%%%%%%%%%%%%%%%%%%%%%%%%%%%%%%%%%

\section*{Required Materials}

Lecture materials will be available on the course webpage under \href{https://cd-public.github.io/courses/intro/s-151f22.html}{schedule}.

The textbook is available online at \href{https://willamette.edu/~esroberts/cs151/reader/CS151-Reader.pdf}{this link} and additional documentation on the PGL library we use for graphics is available at \href{https://willamette.edu/~esroberts/python/pgldoc/}{this link}


% Third Section %%%%%%%%%%%%%%%%%%%%%%%%%%%%%%%%%%%%%%%%%%%

\section*{Prerequisites}

This class is meant to open to all students.

\section*{Accessability}

I will make every effort to ensure all coursework and materials are accessible to all students, including working with on-campus specialists. However, there is always room for improvement. I always appreciate hearing from students about how I can make the course more accessible, so please reach out if there is something I can be doing better!

% Fourth Section %%%%%%%%%%%%%%%%%%%%%%%%%%%%%%%%%%%%%%%%%%%

\section*{Course Objectives}
To gain the skills, knowledge, and confidence necessary to write, test, and debug Python programs requiring several hundred lines of code.

Doing so will require that students be able to:
\begin{itemize}
\item implement algorithms,
\item decompose complex problems,
\item use recursion, 
\item design programming structures,
\item do each of: design, implement, test and debug,  
\item argue that computer science is more than programming.
\end{itemize}
Competency of the techniques of scientific inquiry and programming taught in this course will prepare you to study higher level areas where computation is applied to fundamental problems in the arts, sciences, and humanities.

% Fifth Section %%%%%%%%%%%%%%%%%%%%%%%%%%%%%%%%%%%%%%%%%%%

\section*{Course Structure}

The course will be composed of lectures, labs, homeworks, midterms, and a final project.

\subsection*{Lecture Structure}

Lectures are scheduled for Mondays, Wednesdays, and Fridays at 9:10 AM in "Salem TBA". The schedule
of lectures will be available on the course webpage under \href{https://cd-public.github.io/courses/intro/151f22.html}{schedule}.

\bigskip
\noindent Lectures will be primarily on the white board, with some coding. I will support a remote option
for all lectures, likely through a streaming platform or Zoom, and will post recordings of lectures
on the course website. While I will make every effort to follow best practices for accessible teaching, I will make mistakes! Please, if you find some material is inaccessible for any reason
do not hesitate to reach out. I am currently working to develop better ways from my writing to visible on Zoom, and please let me know if you have any ideas!

\subsection*{Sections}

In addition to three days of lecture, there will also be weekly meetings in smaller groups led by student peer mentors called sections. This sections will prepare you to apply the material taught in class to the assignments.

\subsection*{Homework Structure}

Feedback will be provided on homework assignments. Homework assignments will be considered when determining grades for this course.

\bigskip
\noindent Homeworks will consist of programming assignments to be completed outside of class and submitted
for feedback. There will be be both Problem Sets, which are meant to cover specific techniques from class,
and Projects, which are meant to cultivate the skills of assembly larger programs.

\bigskip
\noindent Homework assignment will be due at 9:10 AM on Fridays so we can discuss them in class on the 
day they are due.

\bigskip
\noindent As a rule, I encourage students to submit their assignment as-is at the due date and not to
submit late work. By way of explanation, it is my experience as both a student and a grader that time spent working on assignments after their deadline is often better spent working on the next assignment.

\bigskip
\noindent As-is submission is supported in the grading policy by dropping the lowest homework
grade. In special circumstances such as extended medical problems or other unforeseeable
emergencies, please reach out and so we can collaboratively develop a more personal
solution to achieve the learning objectives of the course.

\subsubsection*{Midterm Structure}

Feedback will be provided on midterm exams.  Midterm exams will be considered when determining grades for this course.

\bigskip
\noindent There will be two written midterm final exams, intended to be completed individual
without access to notes or documentation. The midterms are intended to achieve a learning
focus of reasoning about computing in isolation from coding environments,
as well provide me as an instructor with greater insight into how effective course
instruction has been.

\subsubsection*{Final Project}

Feedback will be provided on the final project. The final project will be considered when determining grades for this course.

There will be a final project that will be similar in format but distinct in content from
homeworks. Whereas homeworks will including programming assignments for which there is
some correct answer that may be measured against an answer key and notions of coding
style, the project will be a student-centered exploration of computation meant to
give you an opportunity to apply what you've learned in the course while still receiving
support from an instructor.

\bigskip
\noindent The final project will be released as soon as the final homework is due, and will be due at the time of the final examination as set 
by the registrar.

\subsection*{Feedback and Grading}
Feedback will be provided on assignments, midterms, and the final project using a 100 point scale.
This 100 point scale is intended to be familiar to established grading standards, such as letter grades. To provide aggregate feedback for the whole course, these feedback scores will be combined as follows:
\begin{itemize}
	\item \underline{\textbf{40\%}} of your grade will be determined by homework assignments
	\subitem \underline{\textbf{20\%}} each for problem sets and projects
	\item \underline{\textbf{40\%}} of your grade will be determined by midterm exams.
	\subitem \underline{\textbf{20\%}} each for the two midterms.
	\item \underline{\textbf{20\%}} of your grade will be determined by the final project.
\end{itemize}

\noindent Feedback scores will constitute the minimum grade on an assignment, but the instructor
may exercise discretion at any time to award a higher grade. For example, a submitted homework
may not use some important algorithmic technique as submitted, but if the student showed familiarity
with this technique on an earlier assignment or exam, the absence of that technique in a specific
case need not be counted against a student in grading, but only noted in feedback. This corresponds to the high level notion of feedback corresponding to how well an assignment reached the intended learning goals, while the overall course grade is meant to indicate that a student is prepared to succeed in latter coursework. Under this model, the final project will offer an opportunity to show familiarity with all content in the course, so a strong final project can ensure a high course grade
for any student, regardless of prior scores on midterms and homeworks.

\input{college}

\end{document}

