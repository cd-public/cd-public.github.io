% Don't touch this %%%%%%%%%%%%%%%%%%%%%%%%%%%%%%%%%%%%%%%%%%%
\documentclass[11pt]{article}
\usepackage{fullpage}
\usepackage[left=1in,top=1in,right=1in,bottom=1in,headheight=3ex,headsep=3ex]{geometry}
\usepackage{graphicx}
\usepackage{float}
\usepackage{quoting}

\setlength{\parindent}{0em}
\setlength{\parskip}{1em}

\newcommand{\blankline}{\quad\pagebreak[2]}
%%%%%%%%%%%%%%%%%%%%%%%%%%%%%%%%%%%%%%%%%%%%%%%%%%%%%%%%%%%%%%

% Modify Course title, instructor name, semester here %%%%%%%%

\title{CS-399: Networks \& Systems}
\author{Calvin Deutschbein (they)}
\date{Willamette University, Spring 2022}

%%%%%%%%%%%%%%%%%%%%%%%%%%%%%%%%%%%%%%%%%%%%%%%%%%%%%%%%%%%%%%

% Don't touch this %%%%%%%%%%%%%%%%%%%%%%%%%%%%%%%%%%%%%%%%%%%
\usepackage[sc]{mathpazo}
\linespread{1.05} % Palatino needs more leading (space between lines)
\usepackage[T1]{fontenc}
\usepackage[mmddyyyy]{datetime}% http://ctan.org/pkg/datetime
\usepackage{advdate}% http://ctan.org/pkg/advdate
\newdateformat{syldate}{\twodigit{\THEMONTH}/\twodigit{\THEDAY}}
\newsavebox{\MONDAY}\savebox{\MONDAY}{Mon}% Mon
\newcommand{\week}[1]{%
%  \cleardate{mydate}% Clear date
% \newdate{mydate}{\the\day}{\the\month}{\the\year}% Store date
  \paragraph*{\kern-2ex\quad #1, \syldate{\today} - \AdvanceDate[4]\syldate{\today}:}% Set heading  \quad #1
%  \setbox1=\hbox{\shortdayofweekname{\getdateday{mydate}}{\getdatemonth{mydate}}{\getdateyear{mydate}}}%
  \ifdim\wd1=\wd\MONDAY
    \AdvanceDate[7]
  \else
    \AdvanceDate[7]
  \fi%
}
\usepackage{setspace}
\usepackage{multicol}
%\usepackage{indentfirst}
\usepackage{fancyhdr,lastpage}
\usepackage{url}
\pagestyle{fancy}
\usepackage{hyperref}
\usepackage{lastpage}
\usepackage{amsmath}
\usepackage{layout}

\lhead{}
\chead{}
%%%%%%%%%%%%%%%%%%%%%%%%%%%%%%%%%%%%%%%%%%%%%%%%%%%%%%%%%%%%%%

% Modify header here %%%%%%%%%%%%%%%%%%%%%%%%%%%%%%%%%%%%%%%%%
\rhead{\footnotesize Computer Security}

%%%%%%%%%%%%%%%%%%%%%%%%%%%%%%%%%%%%%%%%%%%%%%%%%%%%%%%%%%%%%%
% Don't touch this %%%%%%%%%%%%%%%%%%%%%%%%%%%%%%%%%%%%%%%%%%%
\lfoot{}
\cfoot{\small \thepage/\pageref*{LastPage}}
\rfoot{}

\usepackage{array, xcolor}
\usepackage{color,hyperref}
\definecolor{clemsonorange}{HTML}{EA6A20}
\hypersetup{colorlinks,breaklinks,linkcolor=clemsonorange,urlcolor=clemsonorange,anchorcolor=clemsonorange,citecolor=black}

\begin{document}

\maketitle

\blankline

\begin{tabular*}{.93\textwidth}{@{\extracolsep{\fill}}lr}

%%%%%%%%%%%%%%%%%%%%%%%%%%%%%%%%%%%%%%%%%%%%%%%%%%%%%%%%%%%%%%

% Modify information %%%%%%%%%%%%%%%%%%%%%%%%%%%%%%%%%%%%%%%%%
E-mail: \texttt{ckdeutschbein@willamette.edu} & Web: \href{https://cd-public.github.io/courses/computer_security/451f21.html}{\tt\bf cd-public.github.io/}  \\

 Office Hours: By Appt. \& TBA  &  Lecture MWF 10:20-11:20 \\

 Office: Zoom, Ford 206 & Lecture Hall: Ford 204 \\
 & \\
\hline
\end{tabular*}

\vspace{5 mm}

% First Section %%%%%%%%%%%%%%%%%%%%%%%%%%%%%%%%%%%%%%%%%%%%

\section*{Course Description}

 Networks \& systems form the boundry between abstractions firmly rooted in language and deeper questions in computing regarding the implementation of thinking machines at engineering and physical levels. This course will prepare computer scientists to reason at and across this abstraction boundry to more fully embrace the power of computation.

Students will learn low level languages of C and assembly, use command line tools to study these languages, use features of the operating system including parallelism and networking, and learn how to make changes to operating systems. 

% Second Section %%%%%%%%%%%%%%%%%%%%%%%%%%%%%%%%%%%%%%%%%%%

\section*{Required Materials}

Required materials will be available on the \href{https://cd-public.github.io/courses/computer_security/399s22.html}{course webpage}.


% Third Section %%%%%%%%%%%%%%%%%%%%%%%%%%%%%%%%%%%%%%%%%%%

\section*{Prerequisites}

This class is meant for computer science students who have completed introductory coursework in programming.  Students should complete CS 141/151 and CS 241 before enrolling in this class.

\section*{Accessability}

I will make every effort to ensure all coursework and materials are accessible to all students, including working with on-campus specialists. However, there is always room for improvement. I always appreciate hearing from students about how I can make the course more accessible, so please reach out if there is something I can be doing better!

% Fourth Section %%%%%%%%%%%%%%%%%%%%%%%%%%%%%%%%%%%%%%%%%%%

\section*{Course Objectives}
This course will teach you techniques for reasoning about information and computing and controlled accesses to these resources. As a survey course of the broad discipline of computer security, it will focus on different abstraction levels, from cryptographic code at a low level to the cultural and economic implications of secure and insecure data access at a high level.
\begin{itemize}
\item You will practice using low level languages of C and assembly that do not fully obscure underlying hardware and operating system design decisions.
\item You will gain experience working with command line tools to better understand computation.
\item You will learn some historical efforts to develop and refine computing systems.
\item You will learn about the constitution technologies that compose networks, including the Internet, and facilitate distributed computing.
\item You will be exposed to state-of-the-art security research specific to hardware designs, including computer processors, as an example of ongoing research efforts.
\end{itemize}
This course will equip you apply notions of computer systems to your other coursework, within computer science as well as within the college, and empower you to be a responsible computer scientist and member of an increasingly computer reliant society.

% Fifth Section %%%%%%%%%%%%%%%%%%%%%%%%%%%%%%%%%%%%%%%%%%%

\section*{Course Structure}

The course will be composed of lectures, labs, homeworks, midterms, and a final project.

\subsection*{Class Structure}

Lectures are scheduled for Mondays, Wednesdays, and Fridays at 12:20 AM in "Salem TBD". The schedule
of lectures will be available on the course webpage under \href{https://cd-public.github.io/courses/computer_security/s-399s22.html}{schedule}.

\subsection*{Homework Structure}

Feedback will be provided on homework assignments. Homework assignments will be considered when determining grades for this course.

Homeworks will consist of programming assignments to be completed outside of class and submitted
for feedback. There will be a ``Homework 0'' at the beginning of the semester to get used to
programming and assignment submission, then two homework assignments throughout the semester.
You will always have at least two weeks to complete homework assignments.

Homework assignment will be due at 10:20 AM on Fridays so we can discuss them in class on the
day they are due.

As a rule, I encourage students to submit their assignment as-is at the due date and not to
submit late work. By way of explanation, it is my experience as both a student and a grader that time spent working on assignments after their deadline is often better spent working on the next assignment.

As-is submission is supported in the grading policy by dropping the lowest homework
grade. In special circumstances such as extended medical problems or other unforeseeable
emergencies, please reach out and so we can collaboratively develop a more personal
solution to achieve the learning objectives of the course.

\subsubsection*{Midterm Structure}

Feedback will be provided on midterm exams.  Midterm exams will be considered when determining grades for this course.

There will be two written midterm final exams, intended to be completed individual
without access to notes or documentation. The midterms are intended to achieve a learning
focus of reasoning about data structures in isolation from coding environments,
as well provide me as an instructor with greater insight into how effective course
instruction has been.

\subsubsection*{Final Project}

Feedback will be provided on the final project. The final project will be considered when determining grades for this course.

The final project will be similar to the homeworks but more comprehensive and extensible in nature.


\subsection*{Feedback and Grading}
Feedback will be provided on assignments, midterms, and the final project using a 100 point scale.
Discussions will be graded by participation.
This 100 point scale is intended to be familiar to established grading standards, such as letter grades. To provide aggregate feedback for the whole course, these feedback scores will be combined as follows:
\begin{itemize}
	\item \underline{\textbf{40\%}} of your grade will be determined by homework assignments
	\item \underline{\textbf{40\%}} of your grade will be determined by midterm exams.
	\subitem \underline{\textbf{20\%}} each for the two midterms.
	\item \underline{\textbf{20\%}} of your grade will be determined by the final project.
\end{itemize}

\noindent Feedback scores will constitute the minimum grade on an assignment, but the instructor
may exercise discretion at any time to award a higher grade. For example, a submitted homework
may not use some important algorithmic technique as submitted, but if the student showed familiarity
with this technique on an earlier assignment or exam, the absence of that technique in a specific
case need not be counted against a student in grading, but only noted in feedback. This corresponds to the high level notion of feedback corresponding to how well an assignment reached the intended learning goals, while the overall course grade is meant to indicate that a student is prepared to succeed in latter coursework. Under this model, the final project will offer an opportunity to show familiarity with all content in the course, so a strong final project can ensure a high course grade
for any student, regardless of prior scores on midterms and homeworks.

\input{college}

\end{document}

