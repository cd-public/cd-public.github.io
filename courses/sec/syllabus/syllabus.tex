% Don't touch this %%%%%%%%%%%%%%%%%%%%%%%%%%%%%%%%%%%%%%%%%%%
\documentclass[11pt]{article}
\usepackage{fullpage}
\usepackage[left=1in,top=1in,right=1in,bottom=1in,headheight=3ex,headsep=3ex]{geometry}
\usepackage{graphicx}
\usepackage{float}
\usepackage{quoting}

\setlength{\parindent}{0em}
\setlength{\parskip}{1em}

\newcommand{\blankline}{\quad\pagebreak[2]}
%%%%%%%%%%%%%%%%%%%%%%%%%%%%%%%%%%%%%%%%%%%%%%%%%%%%%%%%%%%%%%

% Modify Course title, instructor name, semester here %%%%%%%%

\title{CS-199: Computer Security}
\author{Calvin Deutschbein (they)}
\date{Willamette University, Fall 2022}

%%%%%%%%%%%%%%%%%%%%%%%%%%%%%%%%%%%%%%%%%%%%%%%%%%%%%%%%%%%%%%

% Don't touch this %%%%%%%%%%%%%%%%%%%%%%%%%%%%%%%%%%%%%%%%%%%
\usepackage[sc]{mathpazo}
\linespread{1.05} % Palatino needs more leading (space between lines)
\usepackage[T1]{fontenc}
\usepackage[mmddyyyy]{datetime}% http://ctan.org/pkg/datetime
\usepackage{advdate}% http://ctan.org/pkg/advdate
\newdateformat{syldate}{\twodigit{\THEMONTH}/\twodigit{\THEDAY}}
\newsavebox{\MONDAY}\savebox{\MONDAY}{Mon}% Mon
\newcommand{\week}[1]{%
%  \cleardate{mydate}% Clear date
% \newdate{mydate}{\the\day}{\the\month}{\the\year}% Store date
  \paragraph*{\kern-2ex\quad #1, \syldate{\today} - \AdvanceDate[4]\syldate{\today}:}% Set heading  \quad #1
%  \setbox1=\hbox{\shortdayofweekname{\getdateday{mydate}}{\getdatemonth{mydate}}{\getdateyear{mydate}}}%
  \ifdim\wd1=\wd\MONDAY
    \AdvanceDate[7]
  \else
    \AdvanceDate[7]
  \fi%
}
\usepackage{setspace}
\usepackage{multicol}
%\usepackage{indentfirst}
\usepackage{fancyhdr,lastpage}
\usepackage{url}
\pagestyle{fancy}
\usepackage{hyperref}
\usepackage{lastpage}
\usepackage{amsmath}
\usepackage{layout}

\lhead{}
\chead{}
%%%%%%%%%%%%%%%%%%%%%%%%%%%%%%%%%%%%%%%%%%%%%%%%%%%%%%%%%%%%%%

% Modify header here %%%%%%%%%%%%%%%%%%%%%%%%%%%%%%%%%%%%%%%%%
\rhead{\footnotesize Computer Security}

%%%%%%%%%%%%%%%%%%%%%%%%%%%%%%%%%%%%%%%%%%%%%%%%%%%%%%%%%%%%%%
% Don't touch this %%%%%%%%%%%%%%%%%%%%%%%%%%%%%%%%%%%%%%%%%%%
\lfoot{}
\cfoot{\small \thepage/\pageref*{LastPage}}
\rfoot{}

\usepackage{array, xcolor}
\usepackage{color,hyperref}
\definecolor{clemsonorange}{HTML}{EA6A20}
\hypersetup{colorlinks,breaklinks,linkcolor=clemsonorange,urlcolor=clemsonorange,anchorcolor=clemsonorange,citecolor=black}

\begin{document}

\maketitle

\blankline

\begin{tabular*}{.93\textwidth}{@{\extracolsep{\fill}}lr}

%%%%%%%%%%%%%%%%%%%%%%%%%%%%%%%%%%%%%%%%%%%%%%%%%%%%%%%%%%%%%%

% Modify information %%%%%%%%%%%%%%%%%%%%%%%%%%%%%%%%%%%%%%%%%
E-mail: \texttt{ckdeutschbein@willamette.edu} & Web: \href{https://cd-public.github.io/courses/computer_security/451f21.html}{\tt\bf cd-public.github.io/}  \\

 Office Hours: By Appt. \& TBA  &  Lecture/Discussion TTh 2:30-4:00 \\

 Office: Zoom, Ford 206 & Lecture Hall: TBD\\
 & \\
\hline
\end{tabular*}

\vspace{5 mm}

% First Section %%%%%%%%%%%%%%%%%%%%%%%%%%%%%%%%%%%%%%%%%%%%

\section*{Course Description}

Cybersecurity can be understood as a mindset or approach rather than a subfield of computer science, such as secure mobile computing, network and operating system security, secure data bases, and secure cryptography algorithms. This course prepares a general audience to incorporate security concepts and ethics into their daily lives and offers some basic familiarity with writing security oriented code. 

% Second Section %%%%%%%%%%%%%%%%%%%%%%%%%%%%%%%%%%%%%%%%%%%

\section*{Required Materials}

Required materials will be available on the course webpage under \href{https://cd-public.github.io/courses/sec/sched.html}{schedule}.


% Third Section %%%%%%%%%%%%%%%%%%%%%%%%%%%%%%%%%%%%%%%%%%%

\section*{Prerequisites}

This class is meant for a general audience.

\section*{Accessability}

I will make every effort to ensure all coursework and materials are accessible to all students, including working with on-campus specialists. However, there is always room for improvement. I always appreciate hearing from students about how I can make the course more accessible, so please reach out if there is something I can be doing better!

% Fourth Section %%%%%%%%%%%%%%%%%%%%%%%%%%%%%%%%%%%%%%%%%%%

\section*{Course Objectives}
This course will teach you techniques for reasoning about information and computing and controlled accesses to these resources. As a survey course of the broad discipline of computer security, it will focus on different abstraction levels, from cryptographic code at a low level to the cultural and economic implications of secure and insecure data access at a high level.
\begin{itemize}
\item You will practice styles of thinking used by security researchers to contextualize their work in the broader context of computing and society.
\item You will gain experience working with common coding practices for security, especially in the context of network and internet security.
\item You will learn some historical efforts to attack and defend various computing systems, and discuss the implications of the state of computer security as a discipline and as deployed in practice..
\item You will learn some theoretical background in formulating notions of security (the ``logical foundations'' of computer security).
\item You will be exposed to state-of-the-art security research specific to hardware designs, including computer processors, as an example of ongoing research efforts.
\end{itemize}
This course will equip you apply notions of computer security to your other coursework, within computer science as well as within the college, and empower you to be a responsible computer scientist and member of an increasingly computer reliant society.

% Fifth Section %%%%%%%%%%%%%%%%%%%%%%%%%%%%%%%%%%%%%%%%%%%

\section*{Course Structure}

The course will be composed of classes, homeworks, midterms, and a final project.

\subsection*{Class Structure}

Classes are scheduled for Tuesdays and Thursdays at 2:30 PM in Ford 204. The schedule
of lectures will be available on the course webpage under \href{https://cd-public.github.io/courses/sec/sched.html}{schedule}.

The course will be composed of two parts. In the first part of the course, class will be used for lecture on security concepts. In the second part of the course, class will be used for discussion of high level implications of computer security.

Lectures will be a combination of white board work and demonstrations. I will support a remote option
for all lectures, likely through Zoom, and will post recordings of lectures
on the course website. While I will make every effort to follow best practices for accessible teaching, I will make mistakes! Please, if you find some material is inaccessible for any reason
do not hesitate to reach out.

Discussions will be conducted in a format to be agreed upon by the class, whether in person, hybrid, or Zoom. Participation in lecture will be required and will be considered in course grades, both during class and in advance.

\subsection*{Homework Structure}

Homeworks will consist of some exploratory exercises using code and written responses to course readings. The precise format of homeworks will be set based on the enrollment.

\subsubsection*{Midterm Structure}

Feedback will be provided on midterm exams.  Midterm exams will be considered when determining grades for this course.

There will be two written midterm exams, intended to be completed as small groups
without access to notes or documentation. The midterms are intended to achieve a learning
focus of reasoning aboutsecurity in isolation from existing resources,
as well provide me as an instructor with greater insight into how effective course
instruction has been.

\subsection*{Discussion Participation}

Discussion participation will be considered when determining grades for this course.

Discussion participation will be intermediate between homework assignments and discussion class
sessions. Discussion class will be driven by reviewing primary or secondary source documents relevant to some prominent case of a security breech of secure systems. Students will be expected to review these materials prior to class, and a submit a 1 or 2 paragraph reflection on the reading no latter than midnight before class, to be submitted by email (to give me time to read them before moderating discussion).

If you will be unable to attend class for any reason, be submit a reflection prior to the deadline and not you will not be able to attend.

\subsubsection*{Final Project}

Feedback will be provided on the final project. The final project will be considered when determining grades for this course.

There will be a final project that will be similar in format but distinct in content from
discussions. Final projects are intended to be undertaken collaboratively with other 
students and provide a detailed overview of an important topics in computer security and 
society.


\subsection*{Feedback and Grading}
Feedback will be provided on assignments, midterms, and the final project using a 100 point scale.
Discussions will be graded by participation.
This 100 point scale is intended to be familiar to established grading standards, such as letter grades. To provide aggregate feedback for the whole course, these feedback scores will be combined as follows:
\begin{itemize}
	\item \underline{\textbf{20\%}} of your grade will be determined by homework assignments
	\subitem \underline{\textbf{10\%}} each for the two highest scored assignments (out of three).
	\item \underline{\textbf{40\%}} of your grade will be determined by midterm exams.
	\subitem \underline{\textbf{20\%}} each for the two midterms.
	\item \underline{\textbf{20\%}} of your grade will be determined by discussion participation.
	\subitem \underline{\textbf{10\%}} each for leading a discussion and participating in other discussions.
	\item \underline{\textbf{20\%}} of your grade will be determined by the final project.
\end{itemize}

\noindent Feedback scores will constitute the minimum grade on an assignment, but the instructor
may exercise discretion at any time to award a higher grade. For example, a submitted homework
may not use some important algorithmic technique as submitted, but if the student showed familiarity
with this technique on an earlier assignment or exam, the absence of that technique in a specific
case need not be counted against a student in grading, but only noted in feedback. This corresponds to the high level notion of feedback corresponding to how well an assignment reached the intended learning goals, while the overall course grade is meant to indicate that a student is prepared to succeed in latter coursework. Under this model, the final project will offer an opportunity to show familiarity with all content in the course, so a strong final project can ensure a high course grade
for any student, regardless of prior scores on midterms and homeworks.

\input{college}

\end{document}

