% Don't touch this %%%%%%%%%%%%%%%%%%%%%%%%%%%%%%%%%%%%%%%%%%%
\documentclass[11pt]{article}
\usepackage{fullpage}
\usepackage[left=1in,top=1in,right=1in,bottom=1in,headheight=3ex,headsep=3ex]{geometry}
\usepackage{graphicx}
\usepackage{float}
\usepackage{quoting}

\setlength{\parindent}{0em}
\setlength{\parskip}{1em}

\newcommand{\blankline}{\quad\pagebreak[2]}
%%%%%%%%%%%%%%%%%%%%%%%%%%%%%%%%%%%%%%%%%%%%%%%%%%%%%%%%%%%%%%

% Modify Course title, instructor name, semester here %%%%%%%%

\title{IDS-101-10: College Colloquium: Scientific Socialism}
\author{Calvin Deutschbein (they) and Nathan Garcia-Diaz (he/they)}
\date{Willamette University, Fall 2022}

%%%%%%%%%%%%%%%%%%%%%%%%%%%%%%%%%%%%%%%%%%%%%%%%%%%%%%%%%%%%%%

% Don't touch this %%%%%%%%%%%%%%%%%%%%%%%%%%%%%%%%%%%%%%%%%%%
\usepackage[sc]{mathpazo}
\linespread{1.05} % Palatino needs more leading (space between lines)
\usepackage[T1]{fontenc}
\usepackage[mmddyyyy]{datetime}% http://ctan.org/pkg/datetime
\usepackage{advdate}% http://ctan.org/pkg/advdate
\newdateformat{syldate}{\twodigit{\THEMONTH}/\twodigit{\THEDAY}}
\newsavebox{\MONDAY}\savebox{\MONDAY}{Mon}% Mon
\newcommand{\week}[1]{%
%  \cleardate{mydate}% Clear date
% \newdate{mydate}{\the\day}{\the\month}{\the\year}% Store date
  \paragraph*{\kern-2ex\quad #1, \syldate{\today} - \AdvanceDate[4]\syldate{\today}:}% Set heading  \quad #1
%  \setbox1=\hbox{\shortdayofweekname{\getdateday{mydate}}{\getdatemonth{mydate}}{\getdateyear{mydate}}}%
  \ifdim\wd1=\wd\MONDAY
    \AdvanceDate[7]
  \else
    \AdvanceDate[7]
  \fi%
}
\usepackage{setspace}
\usepackage{multicol}
%\usepackage{indentfirst}
\usepackage{fancyhdr,lastpage}
\usepackage{url}
\pagestyle{fancy}
\usepackage{hyperref}
\usepackage{lastpage}
\usepackage{amsmath}
\usepackage{layout}

\lhead{}
\chead{}
%%%%%%%%%%%%%%%%%%%%%%%%%%%%%%%%%%%%%%%%%%%%%%%%%%%%%%%%%%%%%%

% Modify header here %%%%%%%%%%%%%%%%%%%%%%%%%%%%%%%%%%%%%%%%%
\rhead{\footnotesize Scientific Socialism}

%%%%%%%%%%%%%%%%%%%%%%%%%%%%%%%%%%%%%%%%%%%%%%%%%%%%%%%%%%%%%%
% Don't touch this %%%%%%%%%%%%%%%%%%%%%%%%%%%%%%%%%%%%%%%%%%%
\lfoot{}
\cfoot{\small \thepage/\pageref*{LastPage}}
\rfoot{}

\usepackage{array, xcolor}
\usepackage{color,hyperref}
\definecolor{clemsonorange}{HTML}{EA6A20}
\hypersetup{colorlinks,breaklinks,linkcolor=clemsonorange,urlcolor=clemsonorange,anchorcolor=clemsonorange,citecolor=black}

\begin{document}

\maketitle

\blankline

\begin{tabular*}{.93\textwidth}{@{\extracolsep{\fill}}lr}

%%%%%%%%%%%%%%%%%%%%%%%%%%%%%%%%%%%%%%%%%%%%%%%%%%%%%%%%%%%%%%

% Modify information %%%%%%%%%%%%%%%%%%%%%%%%%%%%%%%%%%%%%%%%%
E-mail: \texttt{ckdeutschbein@willamette.edu} & Web: \href{https://cd-public.github.io/courses/soc}{\tt\bf cd-public.github.io/}  \\

 Office Hours: By Appt. \& TBA  &  Lecture TTh 12:50-2:20 \\

 Office: Zoom, Ford 206 & Lecture Hall: TBD \\
 & \\
\hline
\end{tabular*}

\vspace{5 mm}

% First Section %%%%%%%%%%%%%%%%%%%%%%%%%%%%%%%%%%%%%%%%%%%%

\section*{Course Description}

What is socialism? What does it mean for socialism to be scientific or utopian? Does socialism create more problems than it solves? The development of socialism is intertwined with cultural and political changes. In this course we will look at the history behind some notable socialist developments and examine their contributions to societies. We will also examine contemporary socialist projects.

\section*{About Us}
Calvin Deutschbein is a %practicing Marxist-Leninist organizer and 
second-year professor of computer science. 
%They moved to Oregon from North Carolina in 2021 after being involved in a number of worker campaigns in the US South, including faculty unionization at Elon University and a number of moderate policing reforms in Chapel Hill.

Nathan Garcia-Diaz is a fourth-year student at Willamette studying Biology and Public Health and colloquia assistant.

% Second Section %%%%%%%%%%%%%%%%%%%%%%%%%%%%%%%%%%%%%%%%%%%

\section*{Required Materials}

Required materials for a given class will be available on the course webpage under \href{https://cd-public.github.io/courses/soc/sched.html}{schedule}.


% Third Section %%%%%%%%%%%%%%%%%%%%%%%%%%%%%%%%%%%%%%%%%%%


\section*{Accessability}

I will make every effort to ensure all coursework and materials are accessible to all students, including working with on-campus specialists. However, there is always room for improvement. I always appreciate hearing from students about how I can make the course more accessible, so please reach out if there is something I can be doing better!

% Fourth Section %%%%%%%%%%%%%%%%%%%%%%%%%%%%%%%%%%%%%%%%%%%

\section*{Course Objectives}
Per the college, we learn and practice techniques to:
\begin{itemize}
\item Critically examine information and/or texts (written, oral, artistic, or quantitative) by identifying central ideas or arguments, making inferences, questioning underlying assumptions, and assessing evidence.
\item Learn to contribute to a constructive classroom climate through participation in thoughtful, informed, and responsive discussion, effective speaking, active listening, and the development of an iterative group process of critical analysis and interpretation.
\item Effectively formulate ideas and arguments, develop them through an iterative process, and express them clearly and persuasively via linguistic, artistic, and/or quantitative modes of communication.
\end{itemize}
Additionally, I will attempt to construct the class in a manner that these techniques specifically to create an inclusive academic environment.

% Fifth Section %%%%%%%%%%%%%%%%%%%%%%%%%%%%%%%%%%%%%%%%%%%

\section*{Course Structure}

The course will be composed of in-class discussions and out-of-class writing.

\subsection*{Class Structure}

Discussions are scheduled for Tuesdays and Thursdays at 12:50 AM. The schedule
of topics will be available on the course webpage under \href{https://cd-public.github.io/courses/soc/sched.html}{schedule}.

Each class will revolve around around a reading for discussion.


\subsection*{Participation}

For each class, each student is to prepare one (1)
compelling question that could be used in class discussion. A compelling question is one that challenges
all of us to think critically about the texts – what they mean, their relation to object/events/relations in our
daily lives, etc. The questions are not simply journalist-type questions about “facts” in the texts
themselves.

In addition, you are to compose a paragraph (at least 150 words) in which you begin to answer your
compelling question. You do not have to answer the question completely; rather demonstrate that you
have thought about where the answer might lie (or, at a minimum, where a discussion of the answer
would begin). The goal is to demonstrate that you are reading closely and critically, and beginning to
synthesize texts and your beliefs in conjunction with the texts. This is an opportunity to dig deep, develop
your capacity to take intellectual and creative risks, and find your academic voice.

Your compelling question and answer should be submitted by email with
"IDS-101" in the subject to \texttt{ckdeutschbein@willamette.edu}.

\subsubsection*{Papers}

There will be both a midterm paper and final paper for this course. 
The midterm paper will be due October 13 at 12:50 PM. 
The final paper will be due November 17 at 12:50 PM.


\subsection*{Feedback and Grading}
Feedback will be provided on papers using a 10 point scale.
Discussions will be graded by participation.
Paper feedback for each paper and discussion participation will all be weighed evenly to determine a letter grade.

\noindent Feedback scores will constitute the minimum grade on an assignment, but the instructor
may exercise discretion at any time to award a higher grade.

\input{college}

\end{document}

